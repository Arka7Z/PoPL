\documentclass[11pt]{article}
\usepackage[dvipsnames]{xcolor}
\usepackage{amsmath,amssymb}
\newcommand{\numpy}{{\tt numpy}}    % tt font for numpy

\topmargin -.5in
\textheight 9in
\oddsidemargin -.25in
\evensidemargin -.25in
\textwidth 7in

\begin{document}

% ========== Edit your name here
\author{Arkajyoti Pal}
\title{Principles of Programming Languages: CS40032}
\maketitle

\medskip

% ========== Begin answering questions here

\begin{enumerate}
    \item \begin{enumerate}
        \item  $(\lambda x.((x z) (\lambda y.(x y))))$
        \item $((\lambda x.(x z)) ( \lambda y.(w (\lambda w.((((w y) z) x))))))$
        \item $(\lambda x.((x y) (\lambda x.(y x))))$
    \end{enumerate}
    \item Free variables are marked in \textcolor{red}{red}.
    \begin{enumerate}
        \item $(\lambda x.((x \textcolor{red}{z}) (\lambda y.(x y))))$
        \item $((\lambda x.(x \textcolor{red}{z})) (\lambda y.(\textcolor{red}{w} (\lambda w.((((w y) \textcolor{red}{z}) \textcolor{red}{x}))))))$
        \item $(\lambda x.((x \textcolor{red}{y}) (\lambda x.(\textcolor{red}{y} x))))$
    \end{enumerate}
    
\item 
\begin{enumerate}
    \item To Prove: not(not true) = true.\\
          \textbf{Proof}: \begin{alignat*}{1}
                 not(not\:true)
                 &=  \lambda x.((x\:false)true)(not\:true) \\
                 &= ((not\:true)\:false)\:true \\
                &=  ((\lambda x.((x\:false) true) true)false) true \\
                 &= (((true\:false) true) false)true\\
                 &= ((((\lambda x. \lambda y.x) false) true)false) true\\
                &= (((\lambda y.false) true) false)true \\
                &= ((false) false) true \\
                &= ((\lambda x.\lambda y.y) false) true\\
                &= (\lambda y.y) true = true 
                \end{alignat*}
    \item To Prove: or false true= true.\\
          \textbf{Proof}:  \begin{alignat*}{2}
                or\:false\:true  &= \lambda x.\lambda y.((x\:true) y) false true \\
                    &= \lambda y.((false\:true) y) true\\
                    &= (false\:true) true\\
                    &= ((\lambda x.\lambda y.y) true) true\\
                    &= (\lambda y.y) true\\
                    &= true\\
            \end{alignat*}
    \item To Prove: SUCC \Bar{2}=\Bar{3}.\\
          \textbf{Proof}:  \begin{alignat*}{3}
                SUCC\:2  &= (\lambda z. \lambda f. \lambda y. f (z f y))(\lambda f. \lambda y. f (f y))\\
                        &=  \lambda f. \lambda y. f ( (\lambda f. \lambda y. f (f y)) f y)\\
                        &= \lambda f. \lambda y. f ( f(f(y)))\\
                    &= 3\\
            \end{alignat*}
    \item To Prove: (Y FACT) 2 = 2\\
    Call FACT as F.\\ $F= \lambda f. \lambda n.\:IF\:n = 0\:THEN\:1\:ELSE\:n\star
(f (n-1))$\\
          \textbf{Proof}:  \begin{alignat*}{4}
                (Y FACT) 2  &=(Y F) 2\\
                &= F (Y F) 2 \\
                &= \lambda f. \lambda n.\:IF\:n =0\:THEN\:1\:ELSE\:n\star(f(n-1)) (Y F) 2\\
        &= \lambda n.\:IF\:n = 0\:THEN\:1\:ELSE\:n\star((YF)(n-1)) 2 \\
        &= \:IF\:2 = 0\:THEN\:1\:ELSE\:2\star ((YF)(2-1))\\
        &= 2\star((YF)1)\\
        &= 2\star(F (YF)1)\\
        &= 2\star( \lambda f. \lambda n.\:IF\:n =0\:THEN\:1\:ELSE\:n\star(f(n-1)) (Y F) 1 )\\
        &= 2\star( \lambda n.\:IF\:n =0\:THEN\:1\:ELSE\:n\star((Y F)(n-1))  1 )\\
        &= 2\star( IF\:1 =0\:THEN\:1\:ELSE\:1\star((Y F)(1-1)) )\\
        &= 2\star(1\star ( (YF) 0  )   )\\
        &= 2\star( (YF) 0)\\
        &= 2\star( F(YF) 0)\\
        &= 2\star( \lambda f. \lambda n.\:IF\:n =0\:THEN\:1\:ELSE\:n\star(f(n-1)) (Y F) 0 )\\
        &=  2\star( \lambda n.\:IF\:n =0\:THEN\:1\:ELSE\:n\star((Y F)(n-1))  0)\\
        &= 2\star( IF\:0 =0\:THEN\:1\:ELSE\:0\star((Y F)(0-1)) )\\
        &= 2\star(1) \\
                    &= 2\\
            \end{alignat*}

\item To Prove: $exp\:\bar{2}\:\bar{3} = \bar{8}$\\
   \textbf{Notation}: $f^n x$ or $f^n(x)$  denotes $n$ applications of $f$ on $x$.\\
          \textbf{Proof}:  \begin{alignat*}{4}
    exp\:\bar{2}\:\bar{3} &= (\lambda mnfx.nm f x)\bar{2}\:\bar{3}\\
    &= \lambda fx. (\lambda fx.f^3 x) (\lambda fx. f^2 x) fx\\
    &= \lambda fx. (\lambda x. (\lambda fx. f^2 x)^3 x) fx\\
    &= \lambda fx. ( \lambda x.   (\lambda fx. f^2 x) ((\lambda fx. f^2 x) ( (\lambda fx. f^2 x) x ))          )fx\\
        &= \lambda fx. ( \lambda x.   (\lambda fy. f^2 y) ((\lambda fz. f^2 z) ( (\lambda fw. f^2 w) x ))          )fx\\
&= \lambda fx. ( \lambda x.   (\lambda fy. f^2 y) ((\lambda fz. f^2 z) ( \lambda w. x^2 w  ))          )fx\\
&= \lambda fx. ( \lambda x.   (\lambda fy. f^2 y) 
      (\lambda z. ( \lambda w. x^2 w  )^2 z)           )fx\\
&= \lambda fx. ( \lambda x.   (\lambda fy. f^2 y) 
      (\lambda z.    ( \lambda w. x^2 w  ) (( \lambda w. x^2 w  ) z)        )           )fx\\
&= \lambda fx. ( \lambda x.   (\lambda fy. f^2 y) 
      (\lambda z.    ( \lambda w. x^2 w  ) ( x^2 z)        )           )fx\\
&= \lambda fx. ( \lambda x.   (\lambda fy. f^2 y) 
      (\lambda z.    (  x^2 ( x^2 z)  )         )           )fx\\
&= \lambda fx. ( \lambda x.   (\lambda fy. f^2 y) 
      (\lambda z.(  x^4 z))           )fx\\
&= \lambda fx. ( \lambda x.   (\lambda y. (\lambda z.(  x^4 z)) ^2 y) 
                )fx\\
&= \lambda fx. ( \lambda x.   (\lambda y. (\lambda z.(  x^4 z))((\lambda z.(  x^4 z)) y)   ) 
                )fx\\
&= \lambda fx. ( \lambda x.   (\lambda y. (\lambda z.(  x^4 z))(x^4 y)   ) 
                )fx\\
&= \lambda fx. ( \lambda x.   (\lambda y. (  x^4 (x^4 y) )  )   )   fx\\
&= \lambda fx.   (\lambda y. (  f^4 (f^4 y) )  ) x\\
&= \lambda fx.   (\lambda y. (  f^8 y )  ) x\\
&= \lambda fx.  f^8 x\\
&= \bar{8}\\
            \end{alignat*} 
    \item 
          \textbf{Solution}:  \begin{alignat*}{5}
                add\:\bar{7}\:\bar{1}  &= (\lambda n.\lambda m.\lambda f.\lambda x. n f (m f x))(\lambda f.\lambda x. f^7 x)(\lambda f.\lambda x. f x)\\
              &= \lambda f.\lambda x. (\lambda f.\lambda x. f^7 x) f ((\lambda f.\lambda x. f x)f x)\\
            &= \lambda f.\lambda x. (\lambda s.\lambda z. s^7 z) f ((\lambda f.\lambda x. f x)f x)\\
            &= \lambda f.\lambda x. (\lambda z. f^7 z)((\lambda g.\lambda y. g y)f x)\\
            &= \lambda f.\lambda x. (\lambda z. f^7 z)(f x)\\
            &= \lambda f.\lambda x. ( f^7 (f x))\\
            &= \lambda f.\lambda x. ( f^8 x)\\
                    &= \bar{8}\\
            \end{alignat*}
            
    \item To Prove: IF FALSE THEN x ELSE y = y\\
    \textbf{Proof}:  \begin{alignat*}{6}
    IF\:FALSE\:THEN\:x\:ELSE\:y &= FALSE\:x\:y  \\
    &= (\lambda x. \lambda y. y) x y\\
    &= ( \lambda y. y) y\\
    &= y\\
    \end{alignat*}
    
        \item To Prove: \textit{add} and \textit{mul} are commutative\\
    \textbf{Proof-I}: First we show that \textit{add}  is commutative, i.e., $add\: \bar{m} \: \bar{n}=add\:\bar{n} \: \bar{m}$ \\
    \begin{alignat*}{7}
    add\:\bar{m}\:\bar{n}  &= (\lambda s.\lambda z.\lambda f.\lambda x. s f (z f x))(\lambda f.\lambda x. f^m x)(\lambda f.\lambda x. f^n x)\\
     &= \lambda f.\lambda x. (\lambda f.\lambda x. f^m x) f ((\lambda f.\lambda x. f^n x)f x)\\
    &= \lambda f.\lambda x. (\lambda s.\lambda z. s^m z) f ((\lambda f.\lambda x. f^n x)f x)\\
    &= \lambda f.\lambda x. (\lambda z. f^m z)((\lambda g.\lambda y. g^n y)f x)\\
    &= \lambda f.\lambda x. (\lambda z. f^m z)(f^n x)\\
    &= \lambda f.\lambda x. ( f^m (f^n x))\\
    &= \lambda f.\lambda x. ( f^{(m+n)} x)\\
                    &= \overline{(m+n)}\\
    \end{alignat*}
        \begin{alignat*}{7}
    add\:\bar{n}\:\bar{m}  &= (\lambda s.\lambda z.\lambda f.\lambda x. s f (z f x))(\lambda f.\lambda x. f^n x)(\lambda f.\lambda x. f^m x)\\
     &= \lambda f.\lambda x. (\lambda f.\lambda x. f^n x) f ((\lambda f.\lambda x. f^m x)f x)\\
    &= \lambda f.\lambda x. (\lambda s.\lambda z. s^n z) f ((\lambda f.\lambda x. f^m x)f x)\\
    &= \lambda f.\lambda x. (\lambda z. f^n z)((\lambda g.\lambda y. g^m y)f x)\\
    &= \lambda f.\lambda x. (\lambda z. f^n z)(f^m x)\\
    &= \lambda f.\lambda x. ( f^n (f^m x))\\
    &= \lambda f.\lambda x. ( f^{(n+m)} x)\\
    &= \lambda f.\lambda x. ( f^{(m+n)} x)\\
                    &= \overline{(m+n)}\\
    \end{alignat*}
    
\textbf{Proof-II}: Secondly, we show that \textit{mul}  is commutative, i.e.,$mul\: \bar{m} \: \bar{n}=mul\:\bar{n} \: \bar{m}$ \\
    \begin{alignat*}{7}
    mul\:\bar{m}\:\bar{n}  &= \\
    &= \overline{(m\star n)}\\
    \end{alignat*}

\end{enumerate}
\end{enumerate}



\end{document}
\grid
\grid
