\documentclass[11pt]{article}
\usepackage[dvipsnames]{xcolor}
\usepackage{amsmath,amssymb}
\usepackage{listings}
\newcommand{\numpy}{{\tt numpy}}    % tt font for numpy

\topmargin -.5in
\textheight 9in
\oddsidemargin -.25in
\evensidemargin -.25in
\textwidth 7in

\begin{document}

% ========== Edit your name here
\author{Arkajyoti Pal}
\title{Principles of Programming Languages: CS40032 \\ Assignment-IV: \textit{Haskell, Scheme and Lisp}}
\maketitle

\medskip

% ========== Begin answering questions here
\lstset{
    keywordstyle=\color{blue}
  , basicstyle=\ttfamily
  , commentstyle={}
  , columns=flexible
  , numbers=left
  , showstringspaces=false
  ,                stringstyle=\color{red},
                commentstyle=\color{magenta},
                morecomment=[l][\color{magenta}]{\#}
  } 

\begin{enumerate}
\item 
( ((Func1 a) b) ( ((Func 2) 4) 5))

\item 
\begin{enumerate}
    \item \textit{Output:} {["tac","god","tna","nep"]}\\
           \textit{Intermediate steps:} map applies the reverse function to each string in the list and combines the reversed strings into a list which is returned as the output.
    \item \textit{Output:} {[[1,7],[1,3,5,3,1,5],[5,7],[1,1,3,1,3,3]]}\\
          \textit{Intermediate steps:} The list  {[1,7,4,8,2]} is added to the front of the list xxs to form a \textit{list of lists}.\\
          xs takes the value of each list within this list of list one by one, x takes on the values within xs(when x takes on the values within the range {[4..8]}, it is expanded to include the numbers from 4 till and including 8) and the odd numbers are selected to form a list of list which is the output.
\end{enumerate}
\item
\begin{lstlisting}[language=Haskell,numbers=none]

insertElement:: a->[a]->Int->[a]
insertElement char (x:xs) 0=char:xs
insertElement char (x:xs) k=x:(insertElement char xs (k-1))

--   Possible function runs are:
--    insertElement 'k' ['a','c','d','e'] 2 -> "acke"
--    insertElement "Frenkie" ["Arthur","Busi","Rakitic"] 2 -> ["Arthur","Busi","Frenkie"]
\end{lstlisting} 
\item 4
\item 5
\item 6

\item 
\begin{lstlisting}[language=Lisp,numbers=none]
( defun multiplyList( list-var k)
         (mapcar #'(lambda(x)(* x k)) list-var ))
(defvar mylist '(1 2 3 4 5))   ;; Set the list initially
(defvar k 2)                   ;; Set the multiplying factor
(print "The initial list is ")
(write mylist)
(print " the multiplying factor is ")
(write k)
(setf mylist (multiplyList mylist k))    ;; Call the function
(print "The multiplied list is ")
(write mylist)
\end{lstlisting}

\item
\begin{lstlisting}[language=Lisp,numbers=none]

(defun fibonacci(n)
    (if (<= n 1) n (+ (fibonacci (- n 1)) (fibonacci (- n 2)) ) )
)
(defvar n 0)
(terpri)
(print "Enter the value of n:")
(terpri)
(setf n (read))
(print "The nth Fibonacci number is: ")
(write (fibonacci n))
\end{lstlisting}

\end{enumerate}


\end{document}
\grid
\grid
