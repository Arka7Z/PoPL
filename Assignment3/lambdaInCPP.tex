\documentclass[11pt]{article}
\usepackage[dvipsnames]{xcolor}
\usepackage{amsmath,amssymb}
\usepackage{listings}
\newcommand{\numpy}{{\tt numpy}}    % tt font for numpy

\topmargin -.5in
\textheight 9in
\oddsidemargin -.25in
\evensidemargin -.25in
\textwidth 7in

\begin{document}

% ========== Edit your name here
\author{Arkajyoti Pal}
\title{Principles of Programming Languages: CS40032 \\ Assignment-III: \textit{$\lambda$ and Functors in C++}}
\maketitle

\medskip

% ========== Begin answering questions here

\lstset{language=C++,
                keywordstyle=\color{blue},
                stringstyle=\color{red},
                commentstyle=\color{green},
                morecomment=[l][\color{magenta}]{\#}
}
\begin{enumerate}
\item 
\begin{enumerate}
\item 
    \begin{lstlisting}
    int age=25;
    auto func=[age](int val){cout<<(age-val)<<endl;};
    \end{lstlisting}
    
\item 
    \begin{lstlisting}
    double var=12.3;
    auto func=[&var](double val)
    {
        var+=1.0;
        return ((int)(var+val));
    };
    \end{lstlisting}
\item   4.3\\
    5.3\\
    5\\
    12
\item The code snippet, given as it is, would give a compilation error. Changing the line 
    \begin{lstlisting}
    cout << l(5) <<   << m(5)<<endl ;
    \end{lstlisting}
to     \begin{lstlisting}
    cout << l(5) <<" "<< m(5)<<endl ;
    \end{lstlisting}
would solve the compilation error and give the following output: \\
13\\
14 13

\item
\begin{lstlisting}
    #include <iostream>
    using namespace std;
    int main() 
    {
        int c=3;
        auto func=[&]()->int
        {   
            ++c; 
            cout<<c;
            return 100.2; 
        };
        func();
        return 0;
    }
\end{lstlisting}
    
\item 4344
\end{enumerate}
\item
\begin{enumerate}
     \item \begin{lstlisting}
#include <bits/stdc++.h>
using namespace std;

class TowerOfHanoi{

    public:
            void operator()(int n, char from, char aux, char to)
            {
                    if(n==1)
                    {
                       cout<<"\t\tMove disc 1 from "<<from<<" to "<<to<<"\n";
                       return;
                    }
                    else
                    {
                       (*this)(n-1,from,to,aux);
                       cout<<"\t\tMove disc "<<n<<" from "<<from<<" to "<<to<<"\n";
                       (*this)(n-1,aux,from,to);
                    }
            }
        
};

int main()
{
    TowerOfHanoi tower;
    tower(8,'A','B','C');
    return 0;
}
     \end{lstlisting}

\item \begin{lstlisting}
         /* Solving Tower of Hanoi puzzle using Lambda Expressions */

#include <iostream>
#include <functional>
using namespace std;

int main()
{
    std::function<void(int,char,char,char)> hanoi;
    hanoi=[&hanoi](int a,char from,char aux,char to)  {
        if(a==1){
           cout<<"\t\tMove disc 1 from "<<from<<" to "<<to<<"\n";
           return;
        }
        else{
           hanoi(a-1,from,to,aux);
           cout<<"\t\tMove disc "<<a<<" from "<<from<<" to "<<to<<"\n";
           hanoi(a-1,aux,from,to);
        }
    };
    cout<<"hanoi(8): "<<endl;
     hanoi(8,'A','B','C');
}
     \end{lstlisting}
 \end{enumerate}
\item \begin{enumerate}
    \item 
    \begin{alignat*}{1}
    (\lambda x.x^2 (\lambda x.(x + 1) 2)))\\
Applicative\:order :\\
&=> (\lambda x.x^2 (\lambda  ​\underline{x} ​.(x + 1) ​\underline{2} ​)))\\
&=> (\lambda  ​\underline{x} ​.x^2 ​\underline{3} ​)\\
&=> 9\\
\end{alignat*}
\item 
\begin{alignat*}{2}
Normal\:order :\\
&=> (\lambda  ​\underline{x} ​.x^2 ​\underline{(\lambda x.(x + 1) 2)}) ​)\\
&=> (\lambda x.(x+1) 2)^2\\
&=> (\lambda  ​\underline{x} ​.(x+1) ​\underline{2} ​) * (\lambda ​\underline{x} ​.(x+1) ​\underline{2} ​)\\
&=> 3 * 3\\
&=> 9
\end{alignat*}
\end{enumerate}
\end{enumerate}



\end{document}
\grid
\grid
